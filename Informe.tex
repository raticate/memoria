\documentclass[12pt,letterpaper]{article}
\usepackage[spanish]{babel}
\usepackage[utf8]{inputenc}
\usepackage{graphicx}
\usepackage{caption}
\usepackage{subcaption}
\usepackage{fancyhdr}
\usepackage{wrapfig}
\usepackage{multicol}
\usepackage{latexsym}
\usepackage{listings}
\usepackage{color}
\usepackage{amsmath}
\usepackage{amsthm}
\usepackage{amsfonts}
\usepackage{setspace}
\usepackage[right=2cm,left=2cm,top=2.5cm,bottom=2cm,headsep=0cm,footskip=0.5cm]{geometry}
%\renewcommand{\baselinestretch}{1.5} % Controla el interlineado el numero indica el factor de aumento

\setlength{\parskip}{\baselineskip} 
%Redefinimos el campo caption como Code
\renewcommand{\lstlistingname}{Código}
%Definir colores:
\definecolor{gray97}{gray}{.97}
\definecolor{gray45}{gray}{.45}
%Presentacion del codigo
\lstset{
	 captionpos=b % Ubicación del Caption 
	 %
	 %Define la apariencia de la caja que contiene el codigo
	 frame=Ltb,
     framerule=0pt,
     aboveskip=0.5cm,
     framextopmargin=3pt,
     framexbottommargin=3pt,
     framexleftmargin=0.4cm,
     framesep=0pt,
     rulesep=.4pt,
     backgroundcolor=\color{gray97},
     rulesepcolor=\color{black},
     %
     %Define los colores a usar      
     showstringspaces = false,
     basicstyle=\small\ttfamily,
	 keywordstyle=\bfseries\color{green!40!black},
  	 commentstyle=\itshape\color{purple!40!black},
  	 identifierstyle=\color{blue},
  	 stringstyle=\color{orange},     
     %commentstyle=\color{gray45},
	 %stringstyle=\ttfamily\color{mymauve},
	 %commentstyle=\color{purple!40!black},
     %keywordstyle=\bfseries\color{blue},
     tabsize=2,%numeros de espacios igual al tab
     %
     numbers=left,
     numbersep=15pt,
     numberstyle=\scriptsize,
     numberfirstline = false,
     breaklines=true,     
     }
%%%%%%%%%%%%%%%%%%%%%%%%%%%%%%%%%%%%%%%%%%%%%%%%%%%%%%%%%%%%%%%%%%%%%%%%%%%
% Comando:
% \encabezado{logo}{factor de escalado}{universidad}{facultad}{departamento}{curso}
% Resultado:
% Encabezado con imagenes
%%%%%%%%%%%%%%%%%%%%%%%%%%%%%%%%%%%%%%%%%%%%%%%%%%%%%%%%%%%%%%%%%%%%%%%%%%%
\newcommand{\encabezado}[6]{
\begin{flushleft}
\begin{figure}[t]%
\includegraphics[scale=#2]{#1}%
\hspace{0.2cm}
\begin{tabular}{l}
\textsc{#3}\\
\textsc{#4}\\
\textsc{#5}\\ 
\textsc{#6}\\
\vspace{.5cm}
\end{tabular}
\end{figure}
\end{flushleft}
}

%%%%%%%%%%%%%%%%%%%%%%%%%%%%%%%%%%%%%%%%%%%%%%%%%%%%%%%%%%%%%%%%%%%%%%%%%%%
% Comando:
% \titulo{titulo}{Codigo}{Empresa}
% Resultado:
% Titulo y subtitulo centrado, nombre al costado inferior izquierdo,
% Sirve para un solo nombre
%%%%%%%%%%%%%%%%%%%%%%%%%%%%%%%%%%%%%%%%%%%%%%%%%%%%%%%%%%%%%%%%%%%%%%%%%%%
\newcommand{\titulo}[2]{
	\begin{center}
 		\Large{\textsc{#1}}\\
 		\vspace{10mm}
 		\Large{\textbf{#2}}\\
 		\vspace{3mm}
	\end{center}
}

%%%%%%%%%%%%%%%%%%%%%%%%%%%%%%%%%%%%%%%%%%%%%%%%%%%%%%%%%%%%%%%%%%%%%%%%%%%
% Comando:
% \datos{Nombre}{Carrera}{Especialidad}{rut}{email}{telefono}{fecha}
% Resultado:
% datos en la parte derecha abajo
%%%%%%%%%%%%%%%%%%%%%%%%%%%%%%%%%%%%%%%%%%%%%%%%%%%%%%%%%%%%%%%%%%%%%%%%%%%
\newcommand{\datos}[8]{
	\begin{flushright}
	\begin{minipage}{0.45\textwidth}
	\begin{flushleft}
			Profesor Guía: #1\\
			Profesor Co-Guía: #2\\
			Alumno: #3 \\
			Carrera: #4\\
			Rut: \texttt{#5}\\
			Email: \texttt{#6}\\
			Telefono: \texttt{#7}\\
			Fecha: #8\\
			\vspace{1cm}
	\end{flushleft}
	\end{minipage}
	\end{flushright}
}

\usepackage[spanish]{babel}
\usepackage{dsfont}
\usepackage{hyperref}
\usepackage[nottoc,numbib]{tocbibind}
\usepackage{todonotes}
\usepackage{enumitem}
%\usepackage{chapterbib}\sectionbib{\section}{section}

%\usepackage[procnames]{listings}
\usepackage{color}
\usepackage{listings}
\definecolor{keywords}{RGB}{255,0,90}
%\definecolor{blue}{rgb}{0,0,113}
\definecolor{red}{RGB}{160,0,0}
\definecolor{green}{rgb}{0,.50,0}

\lstdefinelanguage{Scala}{
  morekeywords={abstract,case,catch,class,def,%
    do,else,extends,false,final,finally,%
    for,if,implicit,import,match,mixin,%
    new,null,object,override,package,%
    private,protected,requires,return,sealed,%
    super,this,throw,trait,true,try,%
    type,val,var,while,with,yield},
  otherkeywords={=>,<-,<\%,<:,>:,\#,@},
  sensitive=true,
  morecomment=[l]{//},
  morecomment=[n]{/*}{*/},
  morestring=[b]",
  morestring=[b]',
  morestring=[b]"""
}

\lstset{language=Scala, 
        basicstyle=\ttfamily\scriptsize, 
        keywordstyle=\color{keywords},
        commentstyle=\color{green},
        stringstyle=\color{red},
        showstringspaces=false,
        identifierstyle=\color{blue}}

% "define" Scala

        
%\pagestyle{fancy}
%\fancyhead[l]{\sf{\large{Informe Práctica I}}} %% Titulo superior Izquierdo.
%\fancyhead[r]{\thepage} %% numero de pagina Derecha.
%\fancyfoot{}
%\renewcommand{\headrulewidth}{0.6pt}
\begin{document}
\singlespacing
%% PORTADA %%
\begin{titlepage}
%% Encabezado %%
	\encabezado{pictures/uchile.png}{.3}{Universidad de Chile}{Facultad de Ciencias Físicas y Matemáticas}{Departamento de Ciencias de la Computación}{}
%% Titulo %%


	\titulo{CC6908: Introducción al Trabajo de Título}{Implementación de un algoritmo distribuido de generación de llaves RSA}
	\vspace{2cm}
%%%%%%%%%%%%%%%%%%%%%%%%%%%%%%%%%%%%%%%%%
% the following lines set up the signature area of the title page  
  \begin{tabbing}
  \mbox{\hspace{95mm}}\= \kill

  \hspace{10mm}  \rule{60mm}{.1mm}            \>        \rule{60mm}{.1mm}\\
    \hspace{26mm}Profesor Guía               	\>        \hspace{14mm}Profesor Co-Guía \\[.6in]
  \hspace{53mm}  \rule{60mm}{.1mm}         \\
  \hspace{67mm}  Alumno Memorista         \\

  \end{tabbing}
%%%%%%%%%%%%%%%%%%%%%%%%%%%%%%%%%%%%%%%%%%
%% Datos %%
%	\vspace{2cm}
	\datos{Javier Bustos}{Alejandro Hevia}{Caterina Muñoz}{Ingeniería Civil en Computación}{18.025.979-1}{caterina@niclabs.cl}{+56984799379}{\today}
\end{titlepage}

\newpage
%% Tabla de contenidos
\spacing{0.75}
\tableofcontents
\newpage
\doublespacing

\section{Introducción}


\section{Motivación}

%Dada la necesidad de optimizar la implementación de los descriptores se eligió restringirse al sistema Android . 
\section{Marco Teórico}
\subsection{Criptografía Asimétrica}
\subsection{Criptografía Umbral}
\subsection{Llaves RSA}

\section{Objetivos}
\subsection{Objetivo general}
El objetivo principal de esta memoria es implementar el algoritmo de generación distribuida de llaves RSA propuesto en \cite{bonehfranklin01}. La razón de la elección de este algoritmo es \todo[inline]{explicar por qué este y no otro} Además, se debe evaluar la implementación de una o más de las optimizaciones propuestas en \cite{bonehfranklin01, fouquestern01} \todo[inline]{buscar otros papers con más optimizaciones}y se deben realizar pruebas para comprobar el correcto funcionamiento del algoritmo implementado.


\subsection{Objetivos específicos}
Para conseguir el objetivo principal de la memoria se pueden detallar varios objetivos específicos expuestos a continuación.

\section{Algoritmo Propuesto}
A continuación se explica el algoritmo propuesto por Boneh y Franklin en \cite{bonehfranklin01}.
Dado que este algoritmo tiene como objetivo la generación distribuida de llaves RSA, al final de su ejecución se debe tener:
\begin{itemize}
\item Un módulo RSA $N = p \cdot q$, donde $p$ y $q$ son primos primos de al menos $n$ bits.
\item Un par de llaves público - privada $e, d$, donde $e \cdot d = 1 \mod \varphi(N)$.
\item $N$ y $e$ son públicos
\item y $d$ está repartido entre los $k$ nodos.
\todo[inline]{falta explicar t}
\end{itemize}
%%
El algoritmo sigue los siguientes pasos:
\begin{enumerate}[label=\textbf{\arabic*})]
\item \textbf{Elección de Candidatos}
\\
Cada uno de los $k$ nodos elige dos números aleatorios de $n$ bits: $q_i$ y $p_i$ que mantiene en secreto. 
\item \textbf{Computación de $N$ usando BGW}
\\
Usando una versioń simplificada del método BGW \cite{BGW}, los nodos calculan $N = (p_1 + p_2 + \ldots + p_k)\cdot(q_1 + q_2 + \ldots + q_k)$. Al final de este paso, $N$ es público.
\item \textbf{Test de Primalidad}
\\
Los nodos ocupan un test de primalidad distribuido para determinar si $N$ es el producto de dos primos. En el caso de que no lo sea, el algoritmo vuelve a empezar.  
\item \textbf{Generación de $d$} 
\\Luego de obtener $N$, los nodos deben computar partes aditivas $d_i$ tal que $\sum_{i=1}^k d_i = d$. Así, al final de este paso, cada nodo queda con una parte $d_i$ de $d$. 
\item \textbf{Repartición de $d$}
\\
Que cada nodo tenga una de las partes $d_i$ generadas en el paso anterior no es suficiente para realizar criptografía umbral. Esto, porque se quiere poder realizar una operación criptográfica con un subconjunto cualquiera de $t$ nodos. Para lograr esto, cada $d_i$ se debe repartir de manera redundante entre los $k$ nodos de manera tal que un subconjunto cualquiera de $t$ nodos pueda ``reconstruir'' el $d_i$ en cuestión. Con esto, se logra que mismo efecto para $d$.  
\end{enumerate}

\section{Diseño del sistema}

\section{Trabajo Realizado}


\section{Plan de trabajo}


\bibliographystyle{plain}
\bibliography{Informe} 

\end{document}